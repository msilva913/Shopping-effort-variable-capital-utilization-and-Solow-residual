% 26-Feb-2024 11:43:53, created by disp_th_moments.m 
 
\begin{center}
\begin{longtable}{lcccccccccccccccccccccc} 
\caption{MATRIX OF CORRELATIONS}\\
 \label{Table:th_corr_matrix}\\
\toprule 
$Variables  $	 & 	 $      Y\_obs$	 & 	 $  Y\_N\_obs$	 & 	 $     SR\_obs$	 & 	 $      I\_obs$	 & 	 $  p\_I\_obs$	 & 	 $      C\_obs$	 & 	 $     NC\_obs$	 & 	 $     NI\_obs$	 & 	 $   util\_obs$	 & 	 $      D\_obs$	 & 	 $      log\_Y$	 & 	 $  log\_Y\_N$	 & 	 $     log\_SR$	 & 	 $      log\_K$	 & 	 $      log\_I$	 & 	 $  log\_p\_I$	 & 	 $      log\_C$	 & 	 $      log\_N$	 & 	 $     log\_NC$	 & 	 $     log\_NI$	 & 	 $   log\_util$	 & 	 $      log\_D$\\
\midrule \endfirsthead 
\caption{(continued)}\\
 \toprule \\ 
$Variables  $	 & 	 $      Y\_obs$	 & 	 $  Y\_N\_obs$	 & 	 $     SR\_obs$	 & 	 $      I\_obs$	 & 	 $  p\_I\_obs$	 & 	 $      C\_obs$	 & 	 $     NC\_obs$	 & 	 $     NI\_obs$	 & 	 $   util\_obs$	 & 	 $      D\_obs$	 & 	 $      log\_Y$	 & 	 $  log\_Y\_N$	 & 	 $     log\_SR$	 & 	 $      log\_K$	 & 	 $      log\_I$	 & 	 $  log\_p\_I$	 & 	 $      log\_C$	 & 	 $      log\_N$	 & 	 $     log\_NC$	 & 	 $     log\_NI$	 & 	 $   log\_util$	 & 	 $      log\_D$\\
\midrule \endhead 
\midrule \multicolumn{23}{r}{(Continued on next page)} \\ \bottomrule \endfoot 
\bottomrule \endlastfoot 
$Y\_obs     $	 & 	       1.0000	 & 	       0.1767	 & 	       0.6486	 & 	       0.8466	 & 	      -0.0663	 & 	       0.9715	 & 	       0.6889	 & 	       0.8197	 & 	       0.9024	 & 	       0.8716	 & 	       0.0067	 & 	      -0.0316	 & 	       0.0657	 & 	      -0.0220	 & 	       0.0081	 & 	      -0.0264	 & 	       0.0063	 & 	       0.0117	 & 	       0.0101	 & 	       0.0189	 & 	       0.0251	 & 	       0.0155 \\ 
$Y\_N\_obs  $	 & 	       0.1767	 & 	       1.0000	 & 	       0.8617	 & 	       0.3144	 & 	       0.0204	 & 	       0.0982	 & 	      -0.5729	 & 	      -0.2533	 & 	      -0.1458	 & 	      -0.0904	 & 	       0.0040	 & 	       0.0369	 & 	       0.0515	 & 	      -0.0073	 & 	       0.0093	 & 	      -0.0068	 & 	       0.0027	 & 	      -0.0003	 & 	      -0.0015	 & 	       0.0051	 & 	       0.0073	 & 	       0.0059 \\ 
$SR\_obs    $	 & 	       0.6486	 & 	       0.8617	 & 	       1.0000	 & 	       0.6762	 & 	      -0.0158	 & 	       0.5735	 & 	      -0.0898	 & 	       0.2268	 & 	       0.3516	 & 	       0.3792	 & 	      -0.0039	 & 	      -0.0013	 & 	       0.0290	 & 	      -0.0162	 & 	      -0.0046	 & 	       0.0100	 & 	      -0.0037	 & 	      -0.0042	 & 	      -0.0052	 & 	       0.0001	 & 	       0.0033	 & 	       0.0023 \\ 
$I\_obs     $	 & 	       0.8466	 & 	       0.3144	 & 	       0.6762	 & 	       1.0000	 & 	      -0.2185	 & 	       0.6963	 & 	       0.4000	 & 	       0.7858	 & 	       0.7217	 & 	       0.7061	 & 	       0.0070	 & 	      -0.0192	 & 	       0.0708	 & 	      -0.0200	 & 	       0.0121	 & 	      -0.0239	 & 	       0.0056	 & 	       0.0104	 & 	       0.0078	 & 	       0.0221	 & 	       0.0236	 & 	       0.0162 \\ 
$p\_I\_obs  $	 & 	      -0.0663	 & 	       0.0204	 & 	      -0.0158	 & 	      -0.2185	 & 	       1.0000	 & 	       0.0080	 & 	      -0.1007	 & 	       0.0111	 & 	       0.0517	 & 	       0.1798	 & 	       0.0059	 & 	       0.0138	 & 	      -0.0178	 & 	       0.0139	 & 	       0.0044	 & 	       0.0308	 & 	       0.0063	 & 	       0.0049	 & 	       0.0048	 & 	       0.0051	 & 	       0.0033	 & 	       0.0074 \\ 
$C\_obs     $	 & 	       0.9715	 & 	       0.0982	 & 	       0.5735	 & 	       0.6963	 & 	       0.0080	 & 	       1.0000	 & 	       0.7509	 & 	       0.7555	 & 	       0.8955	 & 	       0.8609	 & 	       0.0060	 & 	      -0.0340	 & 	       0.0571	 & 	      -0.0207	 & 	       0.0056	 & 	      -0.0250	 & 	       0.0061	 & 	       0.0112	 & 	       0.0102	 & 	       0.0157	 & 	       0.0234	 & 	       0.0136 \\ 
$NC\_obs    $	 & 	       0.6889	 & 	      -0.5729	 & 	      -0.0898	 & 	       0.4000	 & 	      -0.1007	 & 	       0.7509	 & 	       1.0000	 & 	       0.7830	 & 	       0.8237	 & 	       0.7425	 & 	       0.0025	 & 	      -0.0527	 & 	       0.0221	 & 	      -0.0147	 & 	      -0.0002	 & 	      -0.0218	 & 	       0.0032	 & 	       0.0098	 & 	       0.0096	 & 	       0.0107	 & 	       0.0158	 & 	       0.0076 \\ 
$NI\_obs    $	 & 	       0.8197	 & 	      -0.2533	 & 	       0.2268	 & 	       0.7858	 & 	       0.0111	 & 	       0.7555	 & 	       0.7830	 & 	       1.0000	 & 	       0.9063	 & 	       0.8962	 & 	       0.0045	 & 	      -0.0422	 & 	       0.0241	 & 	      -0.0117	 & 	       0.0040	 & 	      -0.0083	 & 	       0.0047	 & 	       0.0107	 & 	       0.0091	 & 	       0.0175	 & 	       0.0179	 & 	       0.0132 \\ 
$util\_obs  $	 & 	       0.9024	 & 	      -0.1458	 & 	       0.3516	 & 	       0.7217	 & 	       0.0517	 & 	       0.8955	 & 	       0.8237	 & 	       0.9063	 & 	       1.0000	 & 	       0.9805	 & 	       0.0016	 & 	      -0.0402	 & 	       0.0262	 & 	      -0.0147	 & 	      -0.0008	 & 	      -0.0033	 & 	       0.0023	 & 	       0.0072	 & 	       0.0061	 & 	       0.0118	 & 	       0.0157	 & 	       0.0108 \\ 
$D\_obs     $	 & 	       0.8716	 & 	      -0.0904	 & 	       0.3792	 & 	       0.7061	 & 	       0.1798	 & 	       0.8609	 & 	       0.7425	 & 	       0.8962	 & 	       0.9805	 & 	       1.0000	 & 	       0.0057	 & 	      -0.0308	 & 	       0.0317	 & 	      -0.0104	 & 	       0.0041	 & 	      -0.0042	 & 	       0.0061	 & 	       0.0105	 & 	       0.0093	 & 	       0.0158	 & 	       0.0194	 & 	       0.0148 \\ 
$log\_Y     $	 & 	       0.0067	 & 	       0.0040	 & 	      -0.0039	 & 	       0.0070	 & 	       0.0059	 & 	       0.0060	 & 	       0.0025	 & 	       0.0045	 & 	       0.0016	 & 	       0.0057	 & 	       1.0000	 & 	       0.9843	 & 	       0.5894	 & 	       0.9746	 & 	       0.9947	 & 	      -0.9043	 & 	       0.9996	 & 	       0.9997	 & 	       0.9996	 & 	       0.9974	 & 	       0.9952	 & 	       0.9998 \\ 
$log\_Y\_N  $	 & 	      -0.0316	 & 	       0.0369	 & 	      -0.0013	 & 	      -0.0192	 & 	       0.0138	 & 	      -0.0340	 & 	      -0.0527	 & 	      -0.0422	 & 	      -0.0402	 & 	      -0.0308	 & 	       0.9843	 & 	       1.0000	 & 	       0.6658	 & 	       0.9401	 & 	       0.9868	 & 	      -0.9362	 & 	       0.9819	 & 	       0.9799	 & 	       0.9795	 & 	       0.9788	 & 	       0.9875	 & 	       0.9824 \\ 
$log\_SR    $	 & 	       0.0657	 & 	       0.0515	 & 	       0.0290	 & 	       0.0708	 & 	      -0.0178	 & 	       0.0571	 & 	       0.0221	 & 	       0.0241	 & 	       0.0262	 & 	       0.0317	 & 	       0.5894	 & 	       0.6658	 & 	       1.0000	 & 	       0.3953	 & 	       0.6615	 & 	      -0.8662	 & 	       0.5695	 & 	       0.5780	 & 	       0.5695	 & 	       0.6145	 & 	       0.6643	 & 	       0.5808 \\ 
$log\_K     $	 & 	      -0.0220	 & 	      -0.0073	 & 	      -0.0162	 & 	      -0.0200	 & 	       0.0139	 & 	      -0.0207	 & 	      -0.0147	 & 	      -0.0117	 & 	      -0.0147	 & 	      -0.0104	 & 	       0.9746	 & 	       0.9401	 & 	       0.3953	 & 	       1.0000	 & 	       0.9484	 & 	      -0.7904	 & 	       0.9798	 & 	       0.9769	 & 	       0.9792	 & 	       0.9634	 & 	       0.9481	 & 	       0.9766 \\ 
$log\_I     $	 & 	       0.0081	 & 	       0.0093	 & 	      -0.0046	 & 	       0.0121	 & 	       0.0044	 & 	       0.0056	 & 	      -0.0002	 & 	       0.0040	 & 	      -0.0008	 & 	       0.0041	 & 	       0.9947	 & 	       0.9868	 & 	       0.6615	 & 	       0.9484	 & 	       1.0000	 & 	      -0.9386	 & 	       0.9916	 & 	       0.9934	 & 	       0.9919	 & 	       0.9976	 & 	       0.9992	 & 	       0.9937 \\ 
$log\_p\_I  $	 & 	      -0.0264	 & 	      -0.0068	 & 	       0.0100	 & 	      -0.0239	 & 	       0.0308	 & 	      -0.0250	 & 	      -0.0218	 & 	      -0.0083	 & 	      -0.0033	 & 	      -0.0042	 & 	      -0.9043	 & 	      -0.9362	 & 	      -0.8662	 & 	      -0.7904	 & 	      -0.9386	 & 	       1.0000	 & 	      -0.8938	 & 	      -0.8980	 & 	      -0.8938	 & 	      -0.9144	 & 	      -0.9387	 & 	      -0.8984 \\ 
$log\_C     $	 & 	       0.0063	 & 	       0.0027	 & 	      -0.0037	 & 	       0.0056	 & 	       0.0063	 & 	       0.0061	 & 	       0.0032	 & 	       0.0047	 & 	       0.0023	 & 	       0.0061	 & 	       0.9996	 & 	       0.9819	 & 	       0.5695	 & 	       0.9798	 & 	       0.9916	 & 	      -0.8938	 & 	       1.0000	 & 	       0.9996	 & 	       0.9999	 & 	       0.9957	 & 	       0.9924	 & 	       0.9996 \\ 
$log\_N     $	 & 	       0.0117	 & 	      -0.0003	 & 	      -0.0042	 & 	       0.0104	 & 	       0.0049	 & 	       0.0112	 & 	       0.0098	 & 	       0.0107	 & 	       0.0072	 & 	       0.0105	 & 	       0.9997	 & 	       0.9799	 & 	       0.5780	 & 	       0.9769	 & 	       0.9934	 & 	      -0.8980	 & 	       0.9996	 & 	       1.0000	 & 	       0.9999	 & 	       0.9976	 & 	       0.9938	 & 	       0.9997 \\ 
$log\_NC    $	 & 	       0.0101	 & 	      -0.0015	 & 	      -0.0052	 & 	       0.0078	 & 	       0.0048	 & 	       0.0102	 & 	       0.0096	 & 	       0.0091	 & 	       0.0061	 & 	       0.0093	 & 	       0.9996	 & 	       0.9795	 & 	       0.5695	 & 	       0.9792	 & 	       0.9919	 & 	      -0.8938	 & 	       0.9999	 & 	       0.9999	 & 	       1.0000	 & 	       0.9964	 & 	       0.9925	 & 	       0.9996 \\ 
$log\_NI    $	 & 	       0.0189	 & 	       0.0051	 & 	       0.0001	 & 	       0.0221	 & 	       0.0051	 & 	       0.0157	 & 	       0.0107	 & 	       0.0175	 & 	       0.0118	 & 	       0.0158	 & 	       0.9974	 & 	       0.9788	 & 	       0.6145	 & 	       0.9634	 & 	       0.9976	 & 	      -0.9144	 & 	       0.9957	 & 	       0.9976	 & 	       0.9964	 & 	       1.0000	 & 	       0.9968	 & 	       0.9974 \\ 
$log\_util  $	 & 	       0.0251	 & 	       0.0073	 & 	       0.0033	 & 	       0.0236	 & 	       0.0033	 & 	       0.0234	 & 	       0.0158	 & 	       0.0179	 & 	       0.0157	 & 	       0.0194	 & 	       0.9952	 & 	       0.9875	 & 	       0.6643	 & 	       0.9481	 & 	       0.9992	 & 	      -0.9387	 & 	       0.9924	 & 	       0.9938	 & 	       0.9925	 & 	       0.9968	 & 	       1.0000	 & 	       0.9942 \\ 
$log\_D     $	 & 	       0.0155	 & 	       0.0059	 & 	       0.0023	 & 	       0.0162	 & 	       0.0074	 & 	       0.0136	 & 	       0.0076	 & 	       0.0132	 & 	       0.0108	 & 	       0.0148	 & 	       0.9998	 & 	       0.9824	 & 	       0.5808	 & 	       0.9766	 & 	       0.9937	 & 	      -0.8984	 & 	       0.9996	 & 	       0.9997	 & 	       0.9996	 & 	       0.9974	 & 	       0.9942	 & 	       1.0000 \\ 
\end{longtable}
 \end{center}
% End of TeX file.
